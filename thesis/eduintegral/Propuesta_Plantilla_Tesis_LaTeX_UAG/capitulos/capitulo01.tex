%Capítulo 1

    El desempeño del estudiante en el día a día en México esta directamente relacionado al nivel de comunicación que tienen los actores principales en el ambiente educativo, los cuales son:
    
    \begin{itemize}
        \item El maestro
        \item El alumno
        \item Los padres de familia
    \end{itemize}
    
    Estos actores tienen que tener una buena comunicación para mantener el desempeño del alumno en un nivel satisfactorio. \\ El desempeño estudiantil en México es medido de manera cuantitativa y cualitativa, se requiere que el alumno tenga una buena conducta lo cuál se mide con la cantidad de reportes que pueda tener o no el alumno, en cuánto a la asistencia a clases, el alumno tiene que asistir al menos un 80\% del total de las clases y finalmente las calificaciones que son númericas marcan si el alumno tiene grado aprobatorio en el ciclo escolar vigente.


    \section{Descripción del problema} \label{descripcionproblema}
    
        Los padres de familia no están involucrados de manera constante en la educación de sus hijos, en consecuencia cuando un padre de familia recibe la información del desempeño de su hijo sucede con un retardo considerable, usualmente cuando ya es demasiado tarde y no hay manera de revertir el problema en el ciclo escolar en el que ocurrio. \\ Hay que remarcar que existen señales que nos indican cuando un alumno está teniendo un mal desempeño academico, esto no sucede de un dia a otro, por lo tanto si estas señales se detectan a tiempo y son resueltas oportunamente el desempeño del alumno puede mantenerse la mayor parte del tiempo favorable. \\ Precisamente, esta problemática es la que se intenta solventar en este proyecto, el problema de que practicamente la comunicación entre los padres de familia y los maestros es muy limitada, lo cual tiene como efecto colateral la afectación del desempeño del alumno.


    \section{Motivación del proyecto} \label{motivacionproyecto}

        La tecnología a crecido colosalmente en las últimas décadas a puntos que jámas hubieramos imaginado hoy en día, este crecimiento se ve reflejado en México y en el mundo, es por ello que esta es una gran época para los ingenieros de software que son capaces de aprovechar las herramientas que ofrecé la industria y el conocimiento generado por la academía para crear cosas que hubieran sido un sueño hace unos años. La motivación principal para desarrollar este proyecto es la de aportar todos los conocimientos que fueron engendrados en mi persona por la universidad autonóma de Guadalajara durante mi maestría y a su vez utilizar toda la capacidad tecnologica con la que contamos hoy en México para ayudar a reducir y/o solventar uno de los problemas que existen en la educación actual.


    \section{Objetivos} \label{objetivos}
    
        Los objetivos de este trabajo buscan mejorar o solventar la problemática de la comunicación entre padres de familia y maestros lo cuál se espera ayuda a mejorar la educación que se imparte en las escuelas de México, generando un nuevo estándar de enseñanza en el país. 


        \subsection{\textbf{\textit{Objetivo General}}}
        
            Mejorar o resolver el problema de comunicación que existe entre los padres de familia y el maestro lo cuál se espera tenga un impacto positivo en el desempeño academico del alumno, esto se realizará mediante el modelado, desarrollo y mejora de una aplicación móvil.


        \subsection{\textbf{\textit{Objetivos específicos}}}
        
            \begin{enumerate}
                \item Mejorar la interacción de los padres de familia con su hijo en el ámbito educativo
                
                \item Mejorar el canal de comunicación directo entre los padres de familia y el maestro con una aplicación móvil
                
                \item Ayudar a facilitar el trabajo del maestro en sus actividades diarias en el salón de clases, como son: tomar asistencias, capturar calificaciones y levantar reportes
                
                \item Notificar a padres de familia los eventos relevantes sobre el desempeño académico y comportamiento de sus hijos mediante el uso de la aplicación móvil
                
                \item Almacenar y resguardar todos los eventos generados por la aplicación móvil para generar una base de conocimiento que nos permita realizar análisis de datos para usarlos en un trabajo futuro que nos ayude a predecir de manera más efectiva los problemas de desempeño
                
                \item Identificar las carácteristicas principales que puedan alertar sobre un posible problema de desempeño sobre uno a varios estudiantes
                
                \item Entregar una arquitectura de software, que pueda ser implementada y replicada sin problemas entre diferentes escuelas simultaneamente
                
                \item Ampliar la cantidad de información que se almacena actualmente, teniendo como punto de referencia el estado actual del sistema educativo en Jalisco

            \end{enumerate}


    \section{Hipótesis} \label{hipotesis}

        Se puede lograr que la comunicación entre los padres de familia y el maestro sea buena si la aplicación móvil que se cree cuenta con las siguientes funcionalidades:

        \begin{itemize} 
        
            \item Notificaciones inmediatas desde la aplicación móvil hacia los padres de familia
            
            \item Que ayude al maestro automatizando algunas de sus actividades rutinarias 
            
            \item Que ponga a disposición de los padres de familia la información del desempeño academico del alumno
            
            \item Que permita alcanzar una educación integral de los alumnos mediante el uso continuo 
            
        \end{itemize}


    \section{Delimitación del proyecto} \label{delimitacionproyecto}

        Gracias a la limitación de tiempo con la que se cuenta se pretende entregar una aplicación móvil 100\% funcional teniendo como fecha limite Diciembre del 2019. Otra limitación que se tiene es la de infraestructura más en concreto, de momento no se cuenta con los recursos suficientes para adquirir el servidor que se necesita para ser utilizado como medio de procesado y almacenamiento de la aplicación móvil. \\ Estas limitaciones a su vez hacen que se tenga que simular el ambiente dónde se desenvolveria el servidor y la aplicación móvil, se pretende que el desempeño sea lo más acercado a la realidad pero se hace el énfasis en que al ser un ambiente virtual será sólo una aproximación más que el desempeño en campo. \\ La última limitación esta en la oportunidad de mejorar los detalles de la aplicación móvil los cuales sólo pueden ser encontrados con la prueba en campo, primero por parte de los maestros, y en segundo lugar ya en el uso a la par de los padres de familia y los maestros.


    \section{Justificación} \label{justificacion}

        Después de un previo análisis del problema se considera que el desarrollo de una aplicación móvil es la solución más adecuada para lograr que el vínculo entre padres de familia y maestros se fortalezca. \\ Esto dado a la conveniencia, facilidad de uso, escalabilidad y rápidez con la cual se puede trabajar utilizando un telefono inteligente hoy en día, a su vez teniendo en cuenta que los telefonos, nos acompañan a todos lados en todo momento lo cual se cree que ayudará a facilitar el uso de la aplicación móvil, lo cual se esperá tenga como efecto que se cumpla con el objetivo del proyecto que es la mejora del desempeño academico del alumno a través del desarrollo y modelado de un canal de comunicación entre padres de familia y maestros. \\ Finalmente, se tiene que considerar que en la actualidad la tecnología para crear dicho canal esta al alcance de estas escuelas a través de nuesros conocimientos.