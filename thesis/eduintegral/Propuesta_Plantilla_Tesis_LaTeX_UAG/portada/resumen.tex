%Resumen

En el ambiente estudiantil existe un reto de mejorar la comunicación que hay entre padres de familia y maestros. Usualmente, cuando el desempeño academico del alumno disminuye los padres de familia son informados una vez que este ya se encuentra bajo, no estan siendo informados en el momento oportuno para prevenir que el desempeño de su hijo baje. Si se mejora la comunicación entre padres de familia y maestro mediante la creación de un canal de comunicación directo e inmediatio esto ayudaria enormemente a los padres de familia a darse cuenta del desempeño del hijo sin tener que esperar hasta ver al maestro. Actualmente las aplicaciones que pueden ser adquiridas en México estan orientadas en la mejora de la labor del maestro, no como tal en la mejora de la comunicación entre padres de familia y maestros para lograr la mejora del desempeño del alumno. Dentro del alcance de este proyecto se propone el modelado y desarrollo de la aplicación móvil Eduintegral \cite{eduintegral}, la cual se pretende sea una poderosa herramienta que sea facilitadora del trabajo realizado por los maestros pero más importante sea el motor para mejorar la comunicación entre los padres de familia y los maestros, a través del uso de notificaciones inmediatas generadas por los maestros hacia los padres de familia. Esto será realizado tomando como partida el estado actual de Eduintegral \cite{eduintegral} y desarrollandolo al punto de que sea una aplicación que pueda ser utilizada en campo, para lo que se mejorará su arquitectura teniendo como enfoque un sistema distribuido, centrado en datos y multiplataforma.