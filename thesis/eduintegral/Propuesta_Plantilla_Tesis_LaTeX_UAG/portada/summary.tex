% Summary

In the scholar environment there is a challenge to improve communication between parents and teachers. Usually, when the student's academic performance decreases parents are informed once their child's performance is already low, they are not being informed in a timely manner to prevent their child's student performance from going down. If communication between parents and teacher is improved by creating a direct and immediate communication channel this would greatly help parents realize their child's performance without having to wait until they go with the teacher. Currently the applications that can be acquired in Mexico are aimed at improving the work of the teacher, not as such in improving communication between parents and teachers to achieve the student performance improvement. Within the scope of this project, the modeling and development of the mobile application Eduintegral \cite{eduintegral} is proposed, which is intended to be a powerful tool that will facilitate the work done by teachers but more important is the engine to improve communication between parents and teachers, through the use of immediate notifications generated by teachers to parents. This will be done taking as a starting point the current state of Eduintegral \cite{eduintegral} and developing it to the point that it is going to be an application that can be used in the field, for which its architecture will be improved taking into account a distributed system, focus on data and multiplatform.